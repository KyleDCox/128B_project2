\documentclass[12pt]{article}
 
\usepackage[margin=1in]{geometry}
\usepackage{amsmath,amsthm,amssymb}
\usepackage{mathtools}
\newcommand{\N}{\mathbb{N}}
\newcommand{\R}{\mathbb{R}}
\newcommand{\Z}{\mathbb{Z}}
\newcommand{\Q}{\mathbb{Q}}

\newenvironment{problem}[2][Problem]{\begin{trivlist}
\item[\hskip \labelsep {\bfseries #1}\hskip \labelsep {\bfseries #2}]}{\end{trivlist}}
\newenvironment{exercise}[2][Exercise]{\begin{trivlist}
\item[\hskip \labelsep {\small\bfseries #1}\hskip \labelsep {\small\bfseries #2}]}{\end{trivlist}}
\newenvironment{epart}[2][Part]{\begin{trivlist}
\item[\hskip \labelsep {\footnotesize\bfseries #1}\hskip \labelsep {\footnotesize\bfseries #2}]}{\end{trivlist}}

\DeclarePairedDelimiter\abs{\lvert}{\rvert}
\DeclarePairedDelimiter\norm{\lVert}{\rVert}

\makeatletter
\let\oldabs\abs
\def\abs{\@ifstar{\oldabs}{\oldabs*}}

\let\oldnorm\norm
\def\norm{\@ifstar{\oldnorm}{\oldnorm*}}
\makeatother

\DeclarePairedDelimiter\ceil{\lceil}{\rceil}
\DeclarePairedDelimiter\floor{\lfloor}{\rfloor}

\newtheorem{theorem}{Theorem}[section]
\newtheorem{corollary}{Corollary}[theorem]
\newtheorem{lemma}[theorem]{Lemma}
 
\begin{document}
 
\title{Project 1}
\author{Steven An, Kyle Cox}
\date{Feburary 21, 2018}
\maketitle

\subsection*{Distribution of work}


\subsection*{Part (i)}
Followed instructions.

\subsection*{Part (ii)}
Done as requested.  
See attached.

\subsection*{Part (iii)}
Let $x$ be $NET$.
We have 
\[OUT = F(x) = \frac{1}{1+e^{-x}}\]
and
\[F^\prime(x) = \frac{e^{-x}}{(1+e^{-x})^2} = \frac{1-1+e^{-x}}{(1+e^{-x})^2} = \frac{1+e^{-x}}{(1+e^{-x})^2} -\frac{1}{(1+e^{-x})^2} = F(x)(1-F(x))\]
which was to be shown.
When $x$ is large, then $F(x)$ will be close to 1 because as $x\rightarrow\infty$, $e^{-x}\rightarrow 0$.
On the other hand, when $x$ small, $F(x)$ will tend to zero.
This is because $x\ll 0$, $e^{-x}$ is very large.
Therefore, as the numerator is constant, $F(x)$ will tend to zero.
Another example of a function that can be used is $-\arctan + \pi/2$.
The function is both differentiable and uniformly bounded.
However, here, when $x$ is negative, the function approaches its positive bound and \textit{vice versa}.
The relation between the input and output need not be standard.
One may consider sine and cosine.
These are bounded and differentiable, but are oscillatory.
For strictly monotonic functions, either $F$ approaches its lower or upper bound at a small value of $x$ and \textit{vice versa}.
\subsection*{Part (iv)}
Only returns neural network's guess.

\subsection*{Part (v)}


\subsection*{Part (vi)}


\subsection*{Part (vii)}



\end{document}